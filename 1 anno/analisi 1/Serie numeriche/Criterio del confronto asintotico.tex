\documentclass[12pt]{article} % Imposta la dimensione del carattere
\usepackage{amsmath}
\usepackage{amssymb}
\usepackage{amsfonts}
\usepackage{lipsum} % Pacchetto per un riempitivo di esempio
\usepackage{geometry} % Per margini più ampi
\geometry{a4paper, margin=1.2in}

\begin{document}

\title{\textbf{Criterio del Confronto Asintotico}}
\author{}
\date{}
\maketitle

\section*{Criterio del Confronto Asintotico}

Se abbiamo $\sum_{n=1}^{\infty} a_n$ e $\sum_{n=1}^{\infty} b_n$, se
\[
\lim_{n \to \infty} \frac{a_n}{b_n} = L \neq 0,
\]
allora le due serie hanno lo stesso carattere. Consideriamo $b_n$ una serie di cui è noto il carattere.

\subsection*{Esempi}

\begin{enumerate}
    \item \textbf{Esempio 1}: \\
    Consideriamo la serie $\sum_{n=1}^{\infty} \frac{1}{2n + 5}$. Proviamo a confrontarla con $\sum_{n=1}^{\infty} \frac{1}{n^2}$ calcolando il limite
    \[
    \lim_{n \to \infty} \frac{\frac{1}{2n+5}}{\frac{1}{n^2}} = +\infty,
    \]
    che non è un numero finito $L$. Proviamo invece a confrontarla con una serie armonica, $\sum_{n=1}^{\infty} \frac{1}{n}$:
    \[
    \lim_{n \to \infty} \frac{\frac{1}{2n+5}}{\frac{1}{n}} = \frac{1}{2},
    \]
    che è un numero finito. Poiché la serie armonica diverge, anche la serie $\sum_{n=1}^{\infty} \frac{1}{2n + 5}$ diverge.

    \item \textbf{Esempio 2}: \\
    Consideriamo la serie $\sum_{n=1}^{\infty} \sin\left(\frac{1}{n^3}\right)$ e come $b_n$ una serie armonica generalizzata $\sum_{n=1}^{\infty} \frac{1}{n^\alpha}$, che converge se $\alpha > 1$ e diverge altrimenti. Calcoliamo il limite
    \[
    \lim_{n \to \infty} \frac{\sin\left(\frac{1}{n^3}\right)}{\frac{1}{n^\alpha}} = \lim_{n \to \infty} \frac{n^\alpha}{n^3}.
    \]
    Ponendo $\alpha = 3$, otteniamo $L = 1 \neq 0$, quindi la serie $\sum_{n=1}^{\infty} \sin\left(\frac{1}{n^3}\right)$ ha lo stesso carattere di $\sum_{n=1}^{\infty} \frac{1}{n^3}$, che converge.

\end{enumerate}

\section*{Criterio di Riemann}

Consideriamo la serie $\sum_{n=1}^{\infty} a_n$ con $\alpha \in \mathbb{R}$. Supponiamo che
\[
\lim_{n \to \infty} n^\alpha a_n = l.
\]
Ci sono due casi:

\begin{enumerate}
    \item Se $\alpha > 1$ e $l$ è un numero reale, allora $\sum_{n=1}^{\infty} a_n$ converge.
    \item Se $0 < \alpha < 1$ e $l \neq 0$ o infinito, allora $\sum_{n=1}^{\infty} a_n$ diverge.
\end{enumerate}

\subsection*{Esempi}

\begin{enumerate}
    \item \textbf{Esempio 1}: \\
    Consideriamo $\sum_{n=2}^{\infty} \frac{\ln(n)}{n^2}$. Per risolvere questa serie possiamo usare il criterio di Riemann. Calcoliamo il limite:
    \[
    \lim_{n \to \infty} n^\alpha \frac{\ln(n)}{n^2}.
    \]
    Con $\alpha = 2$, il limite è $+\infty$, mentre con $\alpha = 1$ il limite è $0$. Dobbiamo quindi scegliere un $\alpha$ intermedio, ad esempio $\alpha = \frac{3}{2}$. Risolvendo, otteniamo che con $\alpha > 1$ la serie converge.
\end{enumerate}

\end{document}
